\section{Introduction}
\label{sec:Introduction}
 
This is the template for typesetting LHCb notes and journal papers.
It should be used for any document in LHCb~\cite{Alves:2008zz} that is to be
publicly available. The format should be used for uploading to
preprint servers and only afterwards should specific typesetting
required for journals or conference proceedings be applied. The main
\LaTeX file contains several options as described in the Latex comment
lines.
 
It is expected that these guidelines are implemented for papers already
before they go into the first collaboration wide review.
 
This template also contains the guidelines for how publications and
conference reports should be written.
The symbols defined in \texttt{lhcb-symbols-def.tex} are compatible with
LHCb guidelines.
 
The front page should be adjusted according to what is
written. Default versions are available for papers, conference reports
and analysis notes. Just comment out what you require in the
\texttt{main.tex} file.
 
This directory contains a file called \texttt{Makefile}.
Typing \texttt{make} will apply all Latex and Bibtex commands
in the correct order to produce a pdf file of the document.
The default Latex compiler is pdflatex, which requires figures
to be in pdf format.
To change to plain Latex, edit line 9 of \texttt{Makefile}.
Typing \texttt{make clean} will remove all temporary files generated by (pdf)latex.
 
There is also a PRL template, which is called \texttt{main-prl.tex}.  You need
to have \textsc{REVTeX 4.1} installed~\cite{REVTeX} to compile this. Typing
\texttt{make prl} produces a PRL-style PDF file. Note that this version is not
meant for LHCb-wide circulation, nor for submission to the arXiv. It is just
available to have a look-and-feel of the final PRL version. Typing \texttt{make
 count} will count the words in the main body.
 
 This template now lives on {\tt overleaf} at \url{https://www.overleaf.com/read/hdmcxdrpdszd}. It is temporarily mapped to {\tt svn} during the transition away from {\tt svn}.  Overleaf documents can be {\tt git}-cloned, which is the recommended way of working, or edited using the web interface. The latter permits making suggestions and comments.\todo{Use {\tt $\backslash$todo} to make comments visible in the pdf.}
% * <patrick.koppenburg@cern.ch> 2018-05-29T05:06:09.535Z:
%
% This is a comment.
%
% ^.
